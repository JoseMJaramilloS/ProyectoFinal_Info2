\documentclass{article}
\usepackage[utf8]{inputenc}
\usepackage[spanish]{babel}
\usepackage{listings}
\usepackage{graphicx}
\graphicspath{ {images/} }
\usepackage{cite}

\begin{document}

\begin{titlepage}
    \begin{center}
        \vspace*{1cm}
            
        \Huge
        \textbf{Proyecto final: Modelamiento de clases}
            
        \vspace{0.5cm}
        \LARGE
        Informatica II
            
        \vspace{1.5cm}
            
        \textbf{Víctor Manuel Jiménez García\\
                José Miguel Jaramillo Sánchez}

        \vfill
            
        \vspace{0.8cm}
            
        \Large
        Departamento de Ingeniería Electrónica y Telecomunicaciones\\
        Universidad de Antioquia\\
        Medellín\\
        Abril 5 de 2022
    \end{center}
\end{titlepage}

\tableofcontents

\newpage
\section{Objetivos}\label{objetivos}
\begin{itemize}
    \item Aplicar los conocimientos adquiridos sobre programación orientada a objetos, modelando las de clases y objetos del proyecto final.
    \item Presentar un panorama general del diseño del juego, a través de las propiedades e interacciones de los elementos del juego por medio de sus atributos y características.
    \item Demostrar la importancia y utilidad de la programación por hardware, así como el uso de módulos físicos para optimizar el uso de software en un diseño.
    \item Diseñar un aplicativo en la plataforma de Arduino integrando programación de C++ para solucionar un desafío  propuesto.
\end{itemize}

\section{Descripción}
\textbf{Alien Invasion: Last Hope}\\

En el año 2017, un grupo de científicos envió un mensaje al espacio exterior buscando vida más allá de los horizontes del sistema solar. 5 años después, la deseada respuesta cae sobre el planeta en forma de caos y destrucción, a manos de una invasión alienígena que busca exterminar la raza humana y conquistar la Tierra. Con sus esperanzas casi muertas y al borde de la aniquilación, la humanidad deberá emprender su última maniobra defensiva liderada por nuestro protagonista y héroe. Una hazaña gigante, pues solo es cuestión de tiempo hasta que el planeta sucumba ante el poder de los invasores.\\

El juego consiste en un soldado que se abre paso en el campo de batalla buscando llegar al final del juego para desactivar un dispositivo de destrucción masiva que tiene un tiempo límite hasta explotar. A medida que avanza deberá ir eliminando aliens, y obteniendo bonificaciones y vida. El mapa tendrá plataformas y mientras más cerca está el jugador de la meta más difícil se hará el juego.

\subsection{Modelos físicos a utilizar}
Los modelos físicos a implementar son: movimiento parabólico, movimiento circular, movimiento oscilatorio, colisiones, entre otros que se consideren oportunos.


\section{Clases: atributos y métodos}

\subsection{Personaje principal}
\noindent\textbf{Atributos:} posiciones (x,y), vidas, munición (balas, granadas).\\
\textbf{Métodos:} métodos getter y setter para las posiciones, vidas y munición; métodos de movimiento y salto; métodos necesarios para graficar en la interfaz; disparo; interacción con bonificaciones.

\subsection{Enemigos}
\noindent\textbf{Atributos:} posiciones (x,y), vidas.\\
\textbf{Métodos:} métodos getter y setter para las posiciones y vidas; métodos de movimiento; métodos necesarios para graficar en la interfaz; ataque.

\subsection{Nivel}
\noindent\textbf{Atributos:}\\
\textbf{Métodos:}


\end{document}
