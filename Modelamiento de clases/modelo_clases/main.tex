\documentclass{article}
\usepackage[utf8]{inputenc}
\usepackage[spanish]{babel}
\usepackage{listings}
\usepackage{graphicx}
\graphicspath{ {images/} }
\usepackage{cite}

\begin{document}

\begin{titlepage}
    \begin{center}
        \vspace*{1cm}
            
        \Huge
        \textbf{Proyecto final: Modelamiento de clases}
            
        \vspace{0.5cm}
        \LARGE
        Informatica II
            
        \vspace{1.5cm}
            
        \textbf{Víctor Manuel Jiménez García\\
                José Miguel Jaramillo Sánchez}

        \vfill
            
        \vspace{0.8cm}
            
        \Large
        Departamento de Ingeniería Electrónica y Telecomunicaciones\\
        Universidad de Antioquia\\
        Medellín\\
        Abril 5 de 2022
    \end{center}
\end{titlepage}

\tableofcontents

\newpage
\section{Introducción}
Con el fin de hacer una recopilación de las prácticas realizadas en el curso y conocimientos previos, este proyecto buscará llevar una temática apocalíptica en la cual se abarcará lo anteriormente mencionado, donde se implementará interfaz gráfica, movimientos físicos, manipulación de archivos, etc. Se pondrá a prueba las habilidades de los desarrolladores creando un videojuego intrépido, divertido y desafiante, en el cual se deberá salvar al mundo usando sus habilidades.
\section{Objetivos}\label{objetivos}
\begin{itemize}
    \item Aplicar los conocimientos adquiridos sobre programación orientada a objetos, modelando las de clases y objetos del proyecto final.
    \item Presentar un panorama general del diseño del juego, a través de las propiedades e interacciones de los elementos del juego por medio de sus atributos y características.
    \item Demostrar la importancia y utilidad de la programación por hardware, así como el uso de módulos físicos para optimizar el uso de software en un diseño.
    \item Diseñar un aplicativo en la plataforma de Arduino integrando programación de C++ para solucionar un desafío  propuesto.
\end{itemize}

\section{Descripción}
\subsection{Alien Invasion: Last Hope}

En el año 2017, un grupo de científicos envió un mensaje al espacio exterior buscando vida más allá de los horizontes del sistema solar. 5 años después, la deseada respuesta cae sobre el planeta en forma de caos y destrucción, a manos de una invasión alienígena que busca exterminar la raza humana y conquistar la Tierra. Con sus esperanzas casi muertas y al borde de la aniquilación, la humanidad deberá emprender su última maniobra defensiva liderada por nuestro protagonista y héroe. Una hazaña gigante, pues solo es cuestión de tiempo hasta que el planeta sucumba ante el poder de los invasores.

\subsection{Aspectos del juego y jugabilidad}
El juego consiste en un soldado como personaje principal que se abre paso en el campo de batalla buscando llegar al final del juego para desactivar un dispositivo de destrucción masiva que tiene un tiempo límite hasta explotar. A medida que avanza deberá ir eliminando aliens, y obteniendo bonificaciones y vida generadas aleatoriamente por el mapa o arrojadas por los enemigos muertos. 

El escenario estará conformado por un suelo y plataformas en el aire además de existir huecos o acantilados por los que si el jugador cae, perderá una vida. La generación de enemigos es aleatoria, sin embargo estará limitada a ciertas regiones del mapa de forma que sea coherente para el jugador. En cuanto al movimiento de estos será dependiendo del alien, por ejemplo uno que camine en las plataformas y otro que se desplace por el aire siguiendo un movimiento oscilatorio o circular, además de incluir en algunos de ellos algún tipo de disparo, esto como una forma de aumentar la dificultad a lo largo del juego.\\

El juego dispondrá de un reloj principal, el jugador deberá llegar a la meta antes de que este se agote, de lo contrario la bomba explotara y el juego acabara. Dentro de las bonificaciones estará la oportunidad de obtener segundos extras para el juego.\\ 

En cuanto al personaje principal, podrá moverse y saltar por el mapa y las plataformas, dispondrá de un arma y granadas para eliminar a los aliens, estas últimas seguirán una trayectoria de acuerdo con el movimiento parabólico


\subsection{Modelos físicos a utilizar}
Los modelos físicos a implementar son: movimiento parabólico, movimiento circular, movimiento oscilatorio, colisiones, entre otros que se consideren oportunos.


\section{Clases: atributos y métodos}

\subsection{Personaje principal}
\noindent\textbf{Atributos:} posiciones (x,y), vidas, munición (balas, granadas).\\
\textbf{Métodos:} métodos getter y setter para las posiciones, vidas y munición; métodos de movimiento y salto; métodos necesarios para graficar en la interfaz; disparo; interacción con bonificaciones.

\subsection{Enemigos}
\noindent\textbf{Atributos:} posiciones (x,y), vidas.\\
\textbf{Métodos:} métodos getter y setter para las posiciones y vidas; métodos de movimiento; métodos necesarios para graficar en la interfaz; ataque.

\subsection{Nivel}
\noindent\textbf{Atributos:} Noción de dificultad incrementando la salud y/o aumentando los enemigos en el mapa, limitar las ayudas a medida que se avanza en el juego\\
\textbf{Métodos:} generar enemigos, generar vida, generar munición, generar plataformas

\subsection{Proyectiles y balas}
\noindent\textbf{Atributos:} posiciones (x,y)\\
\textbf{Métodos:} métodos getter y setter para las posiciones, movimiento (de acuerdo a modelos fisicos, la bala ira en linea recta y la granada describira una trayectoria parabolica).

\subsection{Bonificaciones/ Power-ups}
\noindent\textbf{Atributos:} posiciones (x,y), probabilidad de aparicion y tipo de bonificacion (vida,municion u otros)\\
\textbf{Métodos:} métodos getter y setter para las posiciones, movimiento (la bonificación aparecerá aleatoriamente y se moverá por el mapa)

\subsection{Mainwindow}
\noindent\textbf{Atributos:} listas con los objetos del mapa, aliens, bonificaciones, bloques y plataformas; temporizadores; escena\\
\textbf{Métodos:} eventos de tecla; generación del mapa, enemigos y bonificaciones.

\end{document}
